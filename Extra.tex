\documentclass{scrartcl}
%%% Font packages
\usepackage{tgpagella}\setkomafont{disposition}{\rmfamily\bfseries}
\usepackage[T1]{fontenc}
\usepackage[utf8]{inputenc}
%%%
\usepackage{tikz}
%%% Taal & bettere typografie packages
\usepackage[dutch]{babel}
\usepackage[activate={true,nocompatibility},final,tracking=true,kerning=true,spacing=true,factor=1100,stretch=10,shrink=10]{microtype}
\microtypecontext{spacing=nonfrench}
%%% Wiskunde & fysica
\usepackage{amsmath,amssymb,amsthm}
\numberwithin{equation}{section}

%%%%Herdefiniëring van de wortel-teken
\usepackage{letltxmacro}
\makeatletter
\let\oldr@@t\r@@t
\def\r@@t#1#2{%
    \setbox0=\hbox{$\oldr@@t#1{#2\,}$}\dimen0=\ht0
    \advance\dimen0-0.2\ht0
    \setbox2=\hbox{\vrule height\ht0 depth -\dimen0}%
{\box0\lower0.4pt\box2}}
\LetLtxMacro{\oldsqrt}{\sqrt}
\renewcommand*{\sqrt}[2][\ ]{\oldsqrt[#1]{#2} }
\makeatother

\usepackage{siunitx}
\sisetup{output-decimal-marker = {,}}
%%%
\usepackage{graphicx}
\usepackage{multirow,booktabs}
%%% Verwijzingen & hyperlinks
\usepackage{hyperref}
\usepackage[dutch]{cleveref}
%%%%
\KOMAoptions{DIV=calc,BCOR=.75cm, abstract=true}


\usepackage{graphicx}
\usepackage{caption}
\usepackage{subcaption}
\begin{document}
\section{Inleiding}
Bij deze opdrachten is er gebruik gemaakt van model 1 met de oude startwaarden en constante. In deze opdrachten werd aangenomen dat er sprake is van een exponentiële afname. Hierbij geldt dan volgende vergelijking:
\begin{equation}
    h(t) = h(0) \cdot e^{-\lambda t}
\end{equation}
De hoogte neemt in een vaste tijd met een vaste factor af. Bij een gelijk tijdsinterval van \SI{10}{\second}. Met behulp van de modelleringscapaciteiten van \textit{Coach 7} is $h$ op een gelijk tijdsinterval bepaald tot een tijdstip van \SI{120}{\second}. De resultaten van het model zijn weergeven in \cref{tab:1}. 
\begin{table}[ht]
 \caption{Een schematische overzicht van alle data van model 1.}
    \label{tab:1}
    \centering
    \begin{tabular}{*{2}{S[table-format = 3.2]}} 
    \toprule
      {h(\si{\milli\meter})} & {t(\si{\second})}\\
    \midrule
   30 & 0,00 \\ 
   23 & 10,00\\
   18 & 20,00\\ 
   14 & 30,00\\ 
   11 & 40,00\\ 
    86 & 50,00\\ 
    67 & 60,00\\ 
    52 & 70,00\\
    40 & 80,00\\
    32 & 90,00\\
    25 & 100,00\\
    19 & 110,00\\ 
    15 &120,00 \\
    \bottomrule
    \end{tabular}
\end{table}
 De constante $\lambda$ kan worden bepaald aan de hand van \cref{tab:1}, hieruit kan één willekeurig tijdstip worden geselecteerd om de constante te berekenen. 
 \begin{equation}
    h(t) =h(0) \cdot e^{-\lambda t} 
 \end{equation}
De constante $\lambda$ wordt weergeven in \cref{eq:lambda}
\begin{equation}\label{eq:lambda}
-\frac{\ln\left(\frac{h(t)}{h(0)}\right)}{t}=-\frac{\ln\left(\frac{0.233567}{0.30}\right)}{10.00}=-0.0250.  
\end{equation}
Bij een exponentiële afname is de halveringstijd de tijd waarin de hoogte wordt gehalveerd. Mathematische omgeschreven geldt het volgende:
\begin{equation}\label{eq:half}
t_{\frac{1}{2}} = \frac{t \cdot \log(\frac{1}{2})}{\left(\log \left(\frac{h(t)}{h(0)}\right)\right)} = \frac{10.00 \cdot \log(\frac{1}{2})}{\left(\log \left(\frac{0.233567}{0.30}\right)\right)} = \SI{27.69}{\second}
\end{equation}
Met behulp van \textit{Coach 7} is het ook mogelijk om de constante $\lambda$ te berekenen. Dit kan aan de hand worden gedaan met een trendlijn. Bij een exponentiële functie geldt een algemen formule:
\begin{equation}
    f(x) = a \cdot exp(bx) + c
\end{equation}
Uit \textit{Coach 7} is gebleken dat de waarde van $b \approx-0.0250$. Het antwoord lijkt exact te overeenkomen met het berekende $\lambda$ van \cref{eq:lambda}. 










\newpage
De hoeveelheid volume van een vloeistof dat per tijdseenheid door een zekere oppervlakte passeert, wordt het debiet genoemd. Mathematisch omschreven als volgt:
\begin{equation}\label{eq:debiet}
    Q = \frac{dV}{dt}.
\end{equation}
Volume is oppervlakte waardoor het vloeistof maal de hoogte van dit vloeistof. Hieruit volgt dat $Q$ equivalent is aan het volgende:
\begin{equation}\label{eq:debiet_omschreven}
    Q = \frac{dV}{dt} = \frac{dhA}{dt} = A\frac{dh}{dt} = Av.
\end{equation}
In \cref{eq:debiet_omschreven} is $v$ de snelheid van vloeistof in \si{\meter\per\second} door oppervlakte $A$.

Op elke plek in het gevulde gedeelte van het vat is het debiet gelijk. 
\begin{equation}\label{eq:constant_debiet}
    Av = \text{constant}
\end{equation}

Het doorsnee oppervlakte verschilt weliswaar. Zo is bij het gat $A_1 < A_2$. Uit \cref{eq:constant_debiet} volgt dan:
\begin{equation}\label{eq:v_h}
\begin{split}
    A_1\vec{v}_1 = A_2\vec{v}_2\\
    \vec{v}_2 = \frac{A_1}{A_2}\vec{v}_1\\
     \rightarrow v_2 = -\frac{A_1}{A_2}v_1\\
    \end{split}
    \end{equation}
% Eenheden van een grootheid wordt weergeven als volgt: [grootheid]
$[v_{\text{stroom}}]$ is gelijk aan \si{\meter\per\second}. Uit \cref{eq:eenheid} volgt dat $c$ gelijk is aan:
\begin{equation}\label{eq:eenheid}
[c] = \frac{[v_{(\text{stroom})}]}{[h]} = \frac{\si{\meter\per\second}}{\si{\meter}} = \si{\second}^{-1}
\end{equation}
De constante $c$ heeft dus als eenheid $\si{\second}^{-1}$

Elk willekeurig punt uit \ref{tab:1} kan worden geselecteerd om vervolgens de constante, oftewel ($\lambda$) te berekenen:
\begin{equation}%\label{eq:lambda}
h(t) =h(0) \cdot e^{-\lambda t}
\Rightarrow \lambda = -\frac{\ln\left(\frac{h(t)}{h(0)}\right)}{t}.
\end{equation}
Vergelijking \cref{eq:lambda} beschrijft een functie tussen de constante ($\lambda$) en de initiale hoogte $h_i$. In de proef kan met behulp van de verkregen data en algebraïsche omgeschreven vergelijking \cref{eq:lambda} de ($\lambda$) worden berekend. De constante, oftewel ($\lambda$) wordt weergeven in \cref{eq:lambda}:
\begin{equation}
\begin{split}
-\frac{\ln\left(\frac{h(t)}{h(0)}\right)}{t}=-\frac{\ln\left(\frac{0.233567}{0.30}\right)}{10.00}=-0.0250.  
\end{split}
\end{equation}
Het verkregen antwoord uit \cref{eq:lambda} wordt vergeleken met \textit{Coach 7}. Voor het vergelijken van het antwoord wordt ... geanalyseerd en vervolgens met de handeling functie-fit, wordt het functietype: $f(x)=a \cdot exp(bx)+c$ geselecteerd. De optie schatting wordt geselecteerd en vervolgens worden de variabelen $a$ en $c$ respectievelijk vastgezet op $a=0.30$ en $c=0$, tot slot wordt de optie verfijnd. De verkregen waarde van $b$ komt nauwkeurig overeen met de waarde van \cref{eq:lambda}, namelijk $\lambda=-0.0250$.

Bij exponentiële afname is de halveringstijd de tijd waarin de hoogte wordt gehalveerd. Bij groeifactor $e$ kan de halveringstijd $t$ worden berekend door de vergelijking $e^{-\lambda t}$ op te lossen. De halveringstijd wordt weergeven in \cref{eq:half}.
%\begin{equation}\label{eq:half}
%t_{\frac{1}{2}} = \frac{t \cdot \log(\frac{1}{2})}{\left(\log %\left(\frac{h(t)}{h(0)}\right)\right)} = \frac{10.00 \cdot \log(\frac{1}{2})}{\left(\log %\left(\frac{0.233567}{0.30}\right)\right)} = \SI{27.69}{\second}
%\end{equation}
%Voor een exponentiële functie geldt het volgende functietype: $f(x) = a \cdot exp(bx)+c$. De richtingscoëfficiënt kan worden bepaald door twee punten uit de grafiek te selecteren. $a=\frac{\Delta y}{\Delta x}$ oftewel, $a=\frac{\Delta h}{\Delta t}$.  
\begin{align}\label{eq:A}
\begin{split}
v(h) = -\frac{A1}{A2} \cdot c \cdot h \Rightarrow -\frac{A1}{A2} \cdot c =\frac{v(h)}{h(t)}\\
-\frac{A1}{A2} \cdot c = \frac{-0.0075}{0.3} = -0.0250
\end{split}
\end{align}
Het verkregen antwoord van \cref{eq:A} komt exact overeen met de $\lambda$ uit de functie $h(t) = h(0)\cdot e^{\lambda t}$.

\newpage
\section{Bonus}
Als \cref{eq:lambda} gedifferentieerd wordt, leidt het tot de volgende expressie:
    \begin{equation}\label{eq:h_t_afgeleide_exp}
        \frac{dh}{dt} = h(0)\frac{de^{-\lambda t}}{dt} = h(0)e^{-\lambda t}\cdot-\lambda \rightarrow h(t)\cdot-\lambda
    \end{equation}
    \Cref{eq:h_t_afgeleide_exp} is gelijk aan \cref{eq:v_h}.
    \begin{equation}\label{eq:bonus}
        \begin{split}
            -\frac{A_1}{A_2}v_{\text{stroom}} &= h(t)\cdot-\lambda\\
            -\frac{A_1}{A_2}c\cdot h(t) &= h(t)\cdot-\lambda\\
            \frac{A_1}{A_2}c &= \lambda
        \end{split}
    \end{equation}
    Uit \cref{eq:bonus} volgt:
    \begin{equation*}
        h(t) = h(0)e^{-\left(\frac{A_1}{A_2}c\right)t}
    \end{equation*}
    
\end{document}