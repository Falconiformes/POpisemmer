\documentclass{scrartcl}
%%% Font packages
\usepackage{tgpagella}\setkomafont{disposition}{\rmfamily\bfseries}
\usepackage[T1]{fontenc}
\usepackage[utf8]{inputenc}
%%%
\usepackage{tikz}
%%% Taal & bettere typografie packages
\usepackage[dutch]{babel}
\usepackage[activate={true,nocompatibility},final,tracking=true,kerning=true,spacing=true,factor=1100,stretch=10,shrink=10]{microtype}
\microtypecontext{spacing=nonfrench}
%%% Wiskunde & fysica
\usepackage{amsmath,amssymb,amsthm}
\numberwithin{equation}{section}

%%%%Herdefiniëring van de wortel-teken
\usepackage{letltxmacro}
\makeatletter
\let\oldr@@t\r@@t
\def\r@@t#1#2{%
    \setbox0=\hbox{$\oldr@@t#1{#2\,}$}\dimen0=\ht0
    \advance\dimen0-0.2\ht0
    \setbox2=\hbox{\vrule height\ht0 depth -\dimen0}%
{\box0\lower0.4pt\box2}}
\LetLtxMacro{\oldsqrt}{\sqrt}
\renewcommand*{\sqrt}[2][\ ]{\oldsqrt[#1]{#2} }
\makeatother

\usepackage{siunitx}
\sisetup{output-decimal-marker = {,}}
%%%
\usepackage{graphicx}
\usepackage{multirow,booktabs}
%%% Verwijzingen & hyperlinks
\usepackage{hyperref}
\usepackage[dutch]{cleveref}
%%%%
\KOMAoptions{DIV=calc,BCOR=.75cm, abstract=true}


\usepackage{graphicx}
\usepackage{caption}
\usepackage{subcaption}
\begin{document}
\section{Extra opdrachten}
\begin{table}[ht]
 \caption{Een schematische overzicht van alle data van model 1.}
    \label{tab:1}
    \centering
    \begin{tabular}{c c c} 
    \toprule
      h(\si{\milli\meter})&t(\si{\second})\\
    \midrule
   \num{3.0e+2} & 0.00 \\ 
   \num{2.3e+2} & 10.00\\
   \num{1.8e+2} & 20.00\\ 
   \num{1.4e+2} & 30.00\\ 
   \num{1.1e+2} & 40.00\\ 
    86 & 50.00\\ 
    67 & 60.00\\ 
    52 & 70.00\\
    40 & 80.00\\
    32 & 90.00\\
    25 & 100.00\\
    19 & 110.00\\ 
    15 &120.00 \\
    \bottomrule
    \end{tabular}
\end{table}
Elk willekeurig punt uit \ref{tab:1} kan worden geselecteerd om vervolgens de constante, oftewel ($\lambda$) te berekenen:
\begin{equation}\label{eq:lambda}
h(t) =h(0) \cdot e^{-\lambda t}
\Rightarrow \lambda = -\frac{\ln\left(\frac{h(t)}{h(0)}\right)}{t}.
\end{equation}
Vergelijking \ref{eq:lambda} beschrijft een functie tussen de constante ($\lambda$) en de initiale hoogte $h_i$. In de proef kan met behulp van de verkregen data en algebraïsche omgeschreven vergelijking \ref{eq:lambda} de ($\lambda$) worden berekend. De constante, oftewel ($\lambda$) wordt weergeven in \ref{eq:lambda}:
\begin{equation}
\begin{split}
-\frac{\ln\left(\frac{h(t)}{h(0)}\right)}{t}=-\frac{\ln\left(\frac{0.233567}{0.30}\right)}{10.00}=-0.0250.  
\end{split}
\end{equation}
Het verkregen antwoord uit \ref{eq:lambda} wordt vergeleken met \textit{Coach 7}. Voor het vergelijken van het antwoord wordt \ref{fig:first} geanalyseerd en vervolgens met de handeling functie-fit, wordt het functietype: $f(x)=a \cdot exp(bx)+c$ geselecteerd. De optie schatting wordt geselecteerd en vervolgens worden de variabelen $a$ en $c$ respectievelijk vastgezet op $a=0.30$ en $c=0$, tot slot wordt de optie verfijnd. De verkregen waarde van $b$ komt nauwkeurig overeen met de waarde van \ref{eq:lambda}, namelijk $\lambda=-0.0250$.

Bij exponentiële afname is de halveringstijd de tijd waarin de hoeveelheid wordt gehalveerd. Bij groeifactor $e$ kan de halveringstijd $t$ worden berekend door de vergelijking $e^{-\lambda t}$ op te lossen.
\begin{equation}
e^{-\lambda t} = \frac{1}{2} \Rightarrow t =\frac{\ln(\frac{1}{2})}{-0.0250} = \SI{27.691}{\second}.
\end{equation}



De halveringstijd, oftewel $t_{\frac{1}{2}}$, wordt weergeven in \ref{eq:half}.
\begin{equation}\label{eq:half}
 t_{\frac{1}{2}}=\frac{\frac{\log\left(\frac{h(t)}{h(0)}\right)}{\log(\frac{1}{2})}}{t}
\end{equation}
Voor een exponentiële functie geldt het volgende functietype: $f(x) = a \cdot exp(bx)+c$. De richtingscoëfficiënt kan worden bepaald door twee punten uit de grafiek te selecteren. $a=\frac{\Delta y}{\Delta x}$ oftewel, $a=\frac{\Delta h}{\Delta t}$.  

\begin{align}\label{eq:A}
\begin{split}
v(h) = -\frac{A1}{A2} \cdot c \cdot h \Rightarrow -\frac{A1}{A2} \cdot c =\frac{v(h)}{h(t)}\\
-\frac{A1}{A2} \cdot c = \frac{-0.0075}{0.3} = -0.0250
\end{split}
\end{align}
Het verkregen antwoord van \ref{eq:A} komt exact overeen met de $\lambda$ uit de functie $h(t) = h(0)\cdot e^{\lambda t}$.
\newpage
Bij het differentïeren van de functie wordt eerst de ketttingregel toegepast, daarna wordt gesubstitueerd en vervolgens wordt de functie vereenvoudigd. 
\begin{equation}
    h'(t) = \frac{dh}{dt}\left(h(0)\cdot e^{-\left(\frac{A_1}{A_2}\right) \cdot c}\right)=-\frac{e \cdot \frac{A_1 \cdot c}{A_2} \cdot h(0) \cdot c }{A_2}
\end{equation}

%\section{Grafieken}
%\begin{figure}[ht]
    %\centering
    %\begin{subfigure}{0.3\textwidth}
    %\includegraphics[width=\linewidth]{Data.pdf}
    % \caption{De grafiek met $h$ als functie van $t$}
    % \label{fig:first}
    %\%end{subfigure}
    %\begin{subfigure}{0.3\textwidth}
    %\centering
   % \includegraphics[width=\linewidth]{Datas.pdf}
    %\caption{De grafiek met $v(h)$ als functie van $t$.}
  %  \end{subfigure}
 %   \label{fig:second}
%\end{figure}
%De volgenden grafieken zijn gebaseerd op de gegevens uit tabel %\ref{tab:1}.
%\begin{figure}[ht]
 %   \centering
 %%   \includegraphics[scale=0.7]{Data.pdf}
  %  \caption{De grafiek met $h$ als functie van $t$.}
  %  \label{fig:first}
%\end{figure}

%\begin{figure}[ht]
 %   \centering
  %  \includegraphics[scale=0.7]{Datas.pdf}
   % \caption{De grafiek met $v(h)$ als functie van $t$}
    %\label{fig:second}
%\end{figure}
%\Rightarrow \frac{h(t)}{h(0)}=e^{-\lambda t}\\ \Rightarrow ln\left(\frac{h(t)}{h(0)}\right)=ln(e^{-\lambda t}) 
\end{document}