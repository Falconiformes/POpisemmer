\documentclass[numbers=endperiod]{scrartcl}
%%% Font packages
\usepackage{tgpagella}\setkomafont{disposition}{\rmfamily\bfseries}
\usepackage[T1]{fontenc}
\usepackage[utf8]{inputenc}
%%%
\usepackage{tikz}
%%% Taal & bettere typografie packages
\usepackage[dutch]{babel}
\usepackage[activate={true,nocompatibility},final,tracking=true,kerning=true,spacing=true,factor=1100,stretch=10,shrink=10]{microtype}
\microtypecontext{spacing=nonfrench}
%%% Wiskunde & fysica
\usepackage{amsmath,amssymb,amsthm}
\numberwithin{equation}{section}

%%%%Herdefiniëring van de wortel-teken
\usepackage{letltxmacro}
\makeatletter
\let\oldr@@t\r@@t
\def\r@@t#1#2{%
    \setbox0=\hbox{$\oldr@@t#1{#2\,}$}\dimen0=\ht0
    \advance\dimen0-0.2\ht0
    \setbox2=\hbox{\vrule height\ht0 depth -\dimen0}%
{\box0\lower0.4pt\box2}}
\LetLtxMacro{\oldsqrt}{\sqrt}
\renewcommand*{\sqrt}[2][\ ]{\oldsqrt[#1]{#2} }
\makeatother

\usepackage{siunitx}
\sisetup{output-decimal-marker = {,}}
%%%
\usepackage{graphicx}
\usepackage{multirow,booktabs}
%%% Verwijzingen & hyperlinks
\usepackage{hyperref}
\usepackage[dutch]{cleveref}
%%%%
\KOMAoptions{DIV=calc,BCOR=.75cm, abstract=true}


\setlength{\parindent}{0pt}
%%% Front matter
\begin{document}
%Effe een titel gemaakt met geschikte logo...
\begin{titlepage} 
	\newcommand{\HRule}{\rule{\linewidth}{0.5mm}} % \Hrule is gelijk aan een een horizontale lijn met de lengte van een tekstlijn en dikte van een halve milimeter
	
	\center % Centralisatie van de elementen die volgen na deze command
	
	%------------------------------------------------
	%	Titels/subtitels
	%------------------------------------------------
	
	\textsc{\LARGE Lyceum}\\[1.5cm] % Onze school
	
	\textsc{\Large Kinematica}\\[0.5cm] % Hoofdonderwerp 
	
	\textsc{\large Fysica}\\[0.5cm] % Het vak
	
	%------------------------------------------------
	%	Titel
	%------------------------------------------------
	
	\HRule\\[0.4cm] %\\[<lengte>] is afstand tussen de hoofdtitel en lijn(Hrule)
	
	{\huge\bfseries Invloed luchtweerstand op valversnelling}\\[0.4cm] % Hoofdtitel
	
	\HRule\\[1.5cm]
	
	%------------------------------------------------
	%	Schrijvers
	%------------------------------------------------
	
	\begin{minipage}{0.4\textwidth}
		\begin{flushleft} %De tekst begint links
			\large
			\textit{Auteur}\\
			Fidon \textsc{Namani}\\ % Ik
		
		\end{flushleft}
	\end{minipage}
	~ % Dit golfje geeft aan dat deze twee minipagina's nooit onder elkaar mogen komen te staan.
	\begin{minipage}{0.4\textwidth}
		\begin{flushright} %De tekst begint rechts
			\large
			\textit{Beoordelaar}\\
			 \textsc{E. Zijlstra} % Beoordelaar
		\end{flushright}
	\end{minipage}


	
	%------------------------------------------------
	%	Datum
	%------------------------------------------------
	
	\vfill\vfill\vfill % De datumverschijning is 3/4 lengte van top van papier geplaatst.
	
	{\large\today} % Datum
	
	%------------------------------------------------
	%	Logo
	%------------------------------------------------
	
	\vfill\vfill
%	\includegraphics[width=0.7\textwidth]{seesaw}\\[1cm] % Ons fysica logootje
	 
	%----------------------------------------------------------------------------------------
	
	\vfill %De datum wordt voor de zekerheid nog 1/4 deel van de bodem van papier naar boven geduwd
\end{titlepage}
%%

\hrule
\begin{abstract}
    \textit{Doel}:

    \textit{Methode}:

    \textit{Resultaten \& Discussie}:

    \textit{Conclusie}:
\end{abstract}
\hrule
\newpage
\section{Introduction}
Dit is checken of syncing gebeurt vanaf een local repositorsdnasdasldbsay

\section{Formules}

De hoeveelheid volume van een vloeistof dat per tijdseenheid door een zekere oppervlakte passeert, wordt het debiet genoemd. Mathematisch omschreven als volgt:
\begin{equation}\label{eq:debiet}
    Q = \frac{dV}{dt}.
\end{equation}
Volume is oppervlakte waardoor het vloeistof maal de hoogte van dit vloeistof. Hieruit volgt dat $Q$ equivalent is aan het volgende:
\begin{equation}\label{eq:debiet_omschreven}
    Q = \frac{dV}{dt} = \frac{dhA}{dt} = A\frac{dh}{dt} = Av.
\end{equation}
In \cref{eq:debiet_omschreven} is $v$ de snelheid van vloeistof in \si{\meter\per\second} door oppervlakte $A$.

% Eenheden van een grootheid wordt weergeven als volgt: [grootheid]
$[v_{\text{stroom}}]$ is gelijk aan \si{\meter\per\second}. Uit \cref{eq:eenheid} volgt dat $c$ gelijk is aan:
\begin{equation}\label{eq:eenheid}
[c] = \frac{[v_{(\text{stroom})}]}{[h]} = \frac{\si{\meter\per\second}}{\si{\meter}} = \si{\second}^{-1}
\end{equation}
De constante $c$ heeft dus als eenheid $\si{\second}^{-1}$


\newpage
\section{Reflectie}
Ik ben tevreden met de resultaten van ons uitgevoerde onderzoek. Daarnaast
ben ik trots op ons verslag. De gegevens zijn op een professionele
manier in het verslag verwerkt met een mooie strakke lay-out. Ik ben minder
tevreden met de meetfouten die zijn gemaakt. Ondanks dat ik de theorie
goed hebben toegepast. Ik heb in deze PO geleerd om onze kennis in
een praktische opdracht toe te passen, waardoor ik de onderzoeksvragen
heb kunnen beantwoorden.

\newpage
\appendix
\section{Logboek}
\begin{table}[ht]
\centering
\caption{Een logboek met de van week van uitvoering, activiteit, tijdspendering.}
\begin{tabular}{Scc}
\toprule
{Week van uitvoering} & Activiteit & Tijdspendering in (\si{\hour})\\
\midrule
6 & Oriënteren & 1\\
7 & Oriënteren & 1\\
8 & Oriënteren & 1\\
9 & videometing & 2\\
10 & Dataverwerking & 2\\
11 & Dataverwerking & 2\\
12 & Afronding & 1\\
13 & Controle & $\frac{1}{6}$\\
\bottomrule
\end{tabular}
\end{table}
\end{document}
