\documentclass[numbers=endperiod]{scrartcl}
%%% Font packages
\usepackage{tgpagella}\setkomafont{disposition}{\rmfamily\bfseries}
\usepackage[T1]{fontenc}
\usepackage[utf8]{inputenc}
%%%
\usepackage{tikz}
%%% Taal & bettere typografie packages
\usepackage[dutch]{babel}
\usepackage[activate={true,nocompatibility},final,tracking=true,kerning=true,spacing=true,factor=1100,stretch=10,shrink=10]{microtype}
\microtypecontext{spacing=nonfrench}
%%% Wiskunde & fysica
\usepackage{amsmath,amssymb,amsthm}
\numberwithin{equation}{section}

%%%%Herdefiniëring van de wortel-teken
\usepackage{letltxmacro}
\makeatletter
\let\oldr@@t\r@@t
\def\r@@t#1#2{%
    \setbox0=\hbox{$\oldr@@t#1{#2\,}$}\dimen0=\ht0
    \advance\dimen0-0.2\ht0
    \setbox2=\hbox{\vrule height\ht0 depth -\dimen0}%
{\box0\lower0.4pt\box2}}
\LetLtxMacro{\oldsqrt}{\sqrt}
\renewcommand*{\sqrt}[2][\ ]{\oldsqrt[#1]{#2} }
\makeatother

\usepackage{siunitx}
\sisetup{output-decimal-marker = {,}}
%%%
\usepackage{graphicx}
\usepackage{multirow,booktabs}
%%% Verwijzingen & hyperlinks
\usepackage{hyperref}
\usepackage[dutch]{cleveref}
%%%%
\KOMAoptions{DIV=calc,BCOR=.75cm, abstract=true}


%%% Front matter
\title{Modelleren van leeglopen een blik}
\subject{Fysica\\ Vloeistofmechanica}
\author{Fidon Namani\thanks{BRUDDA!}}
\date{\today}

\begin{document}
\maketitle
\begin{abstract}
    \textit{Doel}:

    \textit{Methode}:

    \textit{Resultaten \& Discussie}:

    \textit{Conclusie}:
\end{abstract}
\section{Introduction}
Dit is checken of syncing gebeurt vanaf een local repositorsdnasdasldbsay







\newpage
\section{Reflectie}
Ik ben tevreden met de resultaten van ons uitgevoerde onderzoek. Daarnaast
ben ik trots op ons verslag. De gegevens zijn op een professionele
manier in het verslag verwerkt met een mooie strakke lay-out. Ik ben minder
tevreden met de meetfouten die zijn gemaakt. Ondanks dat ik de theorie
goed hebben toegepast. Ik heb in deze PO geleerd om onze kennis in
een praktische opdracht toe te passen, waardoor ik de onderzoeksvragen
heb kunnen beantwoorden.

\newpage
\appendix
\section{Logboek}
\begin{table}[ht]
\centering
\caption{Een logboek met de van week van uitvoering, activiteit, tijdspendering.}
\begin{tabular}{Scc}
\toprule
{Week van uitvoering} & Activiteit & Tijdspendering in (\si{\hour})\\
\midrule
6 & Oriënteren & 1\\
7 & Oriënteren & 1\\
8 & Oriënteren & 1\\
9 & videometing & 2\\
10 & Dataverwerking & 2\\
11 & Dataverwerking & 2\\
12 & Afronding & 1\\
13 & Controle & $\frac{1}{6}$\\
\bottomrule
\end{tabular}
\end{table}
\end{document}
